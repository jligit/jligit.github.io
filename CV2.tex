\documentclass[margin,line,pifont,palatino,courier]{res}

\usepackage{pifont}
\usepackage[latin1] { inputenc}

%\topmargin .5in
\oddsidemargin -.3in
\evensidemargin -.3in
\textwidth=5.6in
 \textheight=9.0in
%\itemsep=0in
%\parsep=0in
\usepackage{fancyhdr,hyperref}
%\topmargin=0in
%\textheight=8.5in
\pagestyle{fancy}
\renewcommand{\headrulewidth}{0pt}
\fancyhf{}
%\cfoot{\thepage}
%\lfoot{\textit{\footnotesize Research Statement}}
\rfoot{{\footnotesize \thepage}}


\newenvironment{list1}{
  \begin{list}{\ding{113}}{%
      \setlength{\itemsep}{0in}
      \setlength{\parsep}{0in} \setlength{\parskip}{0in}
      \setlength{\topsep}{0in} \setlength{\partopsep}{0in}
      \setlength{\leftmargin}{0.17in}}}{\end{list}}
\newenvironment{list2}{
  \begin{list}{$\bullet$}{%
      \setlength{\itemsep}{0in}
      \setlength{\parsep}{0in} \setlength{\parskip}{0in}
      \setlength{\topsep}{0in} \setlength{\partopsep}{0in}
      \setlength{\leftmargin}{0.2in}}}{\end{list}}

\begin{document}

\name{Junxian Li \vspace*{.1in}}

\begin{resume}

\section{\sc Contact Information}
%\vspace{.05in}
%\begin{tabular}{@{}p{2.75in}p{2in}}
%	Department of Mathematics                        & \hfill{jli135@illinois.edu}\\
%	University of Illinois at Urbana-Champaign  & \hfill{\href{https://math.illinois.edu/~jli135/}{www.math.illinoi.edu/~jli135}}\\
%	1409 W. Green St.            & \\
%	Urbana, Illinois 61801               & \\
%\end{tabular}
%
%\vspace{.05in}
%\begin{tabular}{@{}p{2.2in}p{3in}}
%Mathematisches Institut                    & \hfill{junxian.li@mathematik.uni-goettingen.de}\\
%Georg-August Universit\"at G\"ottingen
% & \hfill{\href{https://jligit.github.io/}{https://jligit.github.io/}}\\
%Bunsenstra\ss e 3-5
%         & \\
%D-37073 G\"ottingen
%             & \\
%Germany   & 

%\end{tabular}

\vspace{.05in}
\begin{tabular}{@{}p{2.5in}p{2.5in}}
Max Planck Institute   for Mathematics                 & \hfill{jli135@mpim-bonn.mpg.de}\\

 & \hfill{\href{https://jligit.github.io/}{https://jligit.github.io/}}\\

Vivatsgasse 7,
53111 Bonn
             & \\
Germany   & 

\end{tabular}

\section{\sc Research Interests}
$L$-functions, Primes,  Exponential sums, Additive Combinatorics\\
Automorphic Forms
% Algebraic Curves, Dynamical systems
\section{\sc Employment} 
{Max Planck Institute for Mathematics}\hfill{Sept 2019--}

\begin{list1}

		\item[] Mentors: Valentin Blomer and Pieter Moree

\end{list1}
{Georg-August Universit\"at G\"ottingen} \hfill{Sept 2018--Aug 2019}

\begin{list1}

		\item[] Mentors: Valentin Blomer and Harald Helfgott


\end{list1}
\section{\sc Education}

{University of Illinois at Urbana-Champaign}\hfill{Sept 2013--Aug 2018}\\
\vspace*{-.1in}
\begin{list1}
\item[] Ph.D.~in Mathematics
\item[] Advisor: Alexandru Zaharescu

\end{list1}

{ Nanjing University}\hfill{Sept 2009--Aug 2013}\\
\vspace*{-.1in}
\begin{list1}
\item[] B. A.~in Mathematics

\end{list1}






\section{\sc Publications}
1. {Zeros of a family of approximations of {H}ecke {$L$}-functions
associated with cusp forms} (with A. Roy and A. Zaharescu), {\it Ramanujan J.} 41(1-3): 391--419, 2016.

2. {Smooth {$L^2$} distances and zeros of approximations of {D}edekind
	zeta functions} (with M. Nastasescu, A. Roy, and A. Zaharescu), {\it Manuscripta Math.} 154(1-2): 195--223, 2017.
	
3. {A lower bound for the least prime in an arithmetic progression} (with K. Pratt and G. Shakan), {\it Q. J. Math.}, 68(3): 729--758,
2017.

4. {Exact evaluation of second moments associated with some families of
	curves over a finite field} (with R. Donepudi and A. Zaharescu), {\it Finite Fields Appl.} 48: 331--355, 2017.
	
5. {On distinct consecutive $r$-differences} (with G. Shakan), {\it J. Number Theory} 199: 363--376, 2019.
% arXiv:1708.03742.	

6. {A local Benford Law for a class of arithmetic sequences} (with Z. Cai and A. J. Hildebrand), {\it Int. J. Number Theory} 15(3): 613--638, 2019.
  %arXiv:1808.01496
  
7. {Value distribution of $L'(\rho)$} (with A. Zaharescu), {{\it J. Math. Anal. Appl.} 480(1): 123400, 24 pp, 2019.}

8. {Leading Digits of Mersenne Numbers} (with Z. Cai, M Faust, A. J. Hildebrand, and Y. Zhang), {\it Exp. Math.} 1-17, 2019.  %arXiv:1712.04425.

9. {Almost Beatty Partitions} (with A. J. Hildebrand, X. Li, and Y. Xie), {{\it J. Integer Seq.} 22(4): Art. 19.4.6, 34 pp, 2019.}
 %arXiv:1809.08690  
 
 10. {The final problem: an identity from Ramanujan's lost notebook} (with B. Berndt and A. Zaharescu), {\it J. Lond. Math. Soc.} 100(2): 568--591, 2019.

11. {A binary quadratic Titchmarsh divisor problem}
{\it Acta Arithmetica} 192(4): 341--361, 2020. %arXiv:1808.00837

12. {Ducci iterates and similar ordering on sets of visible points} (with A. Tamazyan and A. Zaharescu), {{\it Int. J. Number Theory} 16(1): 1--28, 2020.}
 
 13. {The surprising accuracy of Benford's law in mathematics} (with Z. Cai, M. Faust, A. J. Hildebrand and Y. Zhang), {\it Amer. Math. Monthly}  127(3): 217--237, 2020. 
 
14. {Large values of Dirichlet $L$-functions at zeros of a class of $L$-functions} {\it Canad. J. Math.} to appear.

15. {Lower bounds for discrete negative moments of the Riemann zeta function (with W. Heap and J. Zhao)}, arXiv:2003.09368.

16. {Uniform Titchmarsh divisor problems} (with E. Assing and V. Blomer), arXiv:2005.13915. 

17. {Joint value distribution of $L$-functions on the critical line} (with S. Inoue), arXiv:2102.12724.

\section{\sc Conference Proceedings}
1. {On primes in arithmetic progressions}
%\newblock {
%	Building Bridges 3 conference proceedings to appear.}
\newblock{{Automorphic forms and related topics}, 165--167, {\it Contemp. Math.} 732, Amer. Math. Soc., Providence, RI, 2019} 

2. {The Final Problem: A Series Identity from the Lost Notebook} (with B. C. Bruce and A. Zaharescu), {\it George Andrews - 80 Years of Combinatory Analysis}, 2020.

 
 







\section{\sc Honors and Awards}


{Bateman Fellowship in Number Theory} \hfill{Spring 2018}


{On the List of Teachers Ranked as Excellent by their Students} \hfill{Fall 2017}

%\section{\sc Conference Specific Grants}
%
%
%\emph{AMS-MRC Grant for the JMM } \hfill{ \em  Jan 2019}
%
%
%\emph{US Junior Oberwolfach Fellows (NSF Grant)} \hfill{ \em  Oct 2017}
%
%\emph{PCMI-GSS Travel Award } \hfill{ \em  June 2017}
%
%\emph{UIUC-AWM Graduate Travel Funding } \hfill{ \em  2017--2018}
%
%\emph{AMS Graduate Student Travel Grant for the JMM} \hfill{ \em  Jan 2017}


\section{\sc Teaching Experience}
Math 415 Linear Algebra, Instructor {\hfill  UIUC, Fall 2017}\\
Math 415 Linear Algebra, Instructor {\hfill  UIUC, Spring 2017}\\
Math 231 Calculus II, Instructor {\hfill  UIUC, Spring 2016}\\
Math 241 Calculus III, Instructor {\hfill  UIUC, Fall 2016} \\
Math 241 Calculus III, Instructor {\hfill  UIUC, Spring 2015}


\section{\sc Undergraduate \\Mentoring}
\begin{list1}
	\item{Illinois Geometry Lab Graduate Student Mentor}
	\begin{list2}
		\item Almost Beatty Partitions {\hfill  Fall 2018}
		\item Beatty sequences, and Partitions of the Integers{\hfill  Spring 2018}
		\item Chaotic maps and exotic number systems{\hfill  Fall 2017}
		\item Finding integers in group orbits {\hfill  Spring 2017}
		\item Local Benford's Law {\hfill  Fall 2016} 
		\item Leading digit distribution {\hfill  Spring 2016} 
		\item Random Walk in number theory {\hfill  Fall 2015} 
		\item Fractals, Patterns and Randomness in Number Theory {\hfill Spring 2015} 
		\item Fourier Series with Number theoretic coefficients {\hfill  Fall 2014} 
		\item Symmetry in Nature {\hfill Spring 2014}
	\end{list2}
\end{list1}

\section{\sc Professional \\Services}

\begin{list1}
	\item{Organizer of AMS Special Session at the Joint Mathematics Metting} {\hfill  2019}
	
	\begin{list2}
		\item Number Theoretic Methods in Hyperbolic Geometry 
	\end{list2}
	\item{Organizer of Graduate Student Number Theory Seminar in UIUC} {\hfill 2016--2018}
	\item Referee: 
	\begin{list2}
		\item J. Number Theory
		\item Math. Reports
		\item  Rev. Roumaine Math. Pures Appl.
		\item  J. Math. Sci. Adv. Appl.
	\end{list2}
	%\item Membership: American Mathematical Society
	
\end{list1}

\section{\sc Conferences\\ and Seminar Talks}
\begin{list1}
\item{Uniform Titchmarsh Divisor Problems} \\{Japan Europe Number Theory Exchange Seminar}. {\hfill Jan 2021}
\item{ Joint Value Distribution of $L$-functions.} \\{Oberseminar Analytic Number Theory, Bonn(online)}.{\hfill Nov 2020}
\item{ Derivative of the Riemann zeta function at its zeros.} \\{Analytic Number Theory Meeting, IHP (online)}.{\hfill Jun 2020}

\item{Extreme values of $L$-functions} \\{ Number theory lunch seminar, MPIM}.{\hfill Oct 2019}

\item{Extreme values of $L$-functions} \\{Oberseminar analytic number theory, Georg-August Universit\"at G\"ottingen}.{\hfill Nov 2018}
 
 \item{The Unreasonable Effectiveness of Benford's Law in Mathematics}\\Joint with A. J. Hildebrand, Number Theory Seminar, UIUC.{\hfill Apr 2018}
 
\item{Primes in arithmetic progressions} \\{Junior Mathematics Colloquium, Georg-August Universit\"at G\"ottingen}.{\hfill Dec 2017}

\item{Randomness in Number Theory} \\{Graduate Student Colloquium, UIUC}.{\hfill Nov 2017}


\item{Primes in arithmetic progressions} \\{Where Geometry meets Number Theory, a conference in honor of \\the 60th birthday of Per Salberger, Gothenburg}.{\hfill July 2017}
	
 	\item{The least prime in an arithmetic progression} \\
	Joint Mathematics Meeting, Atlanta. {\hfill Jan 2017}

 \item{On the least prime in an arithmetic progression}\\ Number Theory Seminar, UIUC.{\hfill Sept 2016}

\item{A lower bound on the least prime in an arithemetic progression},\\ Workshop on Automorphic Forms and Related Topics, Sarajevo . {\hfill Jul 2016}
 
 \item{Approximations of $L$-functions} \\2015 Midwest Number Theory Conference for Graduate Students \\and Recent Ph. D's. {\hfill Oct 2015}
 
 \item{ Approximations of $L$-functions} \\	Graduate Student Number Theory Seminar, UIUC.{\hfill Nov 2015}
 
 \item{ Bailey Pairs and Bailey chains} \\	$q$-series  Seminar, UIUC.{\hfill Apr 2015}

 \item{ Basic Hypergoemetric functions} \\	$q$-series  Seminar, UIUC.{\hfill Mar 2015}
\end{list1}

\section{\sc Research Experience}
\begin{list1}
\item {Zeta functions, CIRM}{\hfill Dec 2019}

\item {Second Symposium on Analytic Number Theory, Cetraro}{\hfill July 2019}

\item {Rational points on irrational varieties, IHP}{\hfill June 2019}

\item {L-functions and Multiplicative Number Theory, U of Mississippi}{\hfill May 2019}

\item {Distribution of values of zeta functions and L-functions, RIKEN}{\hfill Mar 2019}

\item {Workshop and Winter School on Local Statistics of Point Sequences, Linz}{\hfill Feb 2019}

\item {Building Bridges: 4th EU/US Summer School \\and Workshop on Automorphic Forms and Related Topics
} {\hfill  July 2018}
%\item{Computational aspects of Shimura Curves,\\The resolvent kernel and the Maass-Shimura-Shintani lifting,\\ Theta liftings and modularity of abelian varieties}

\item {Hausdorff School: L-functions: Open Problems and Current Methods}{\hfill  June 2018}

\item {MRC: Number Theoretic Methods
	in Hyperbolic Geometry
	} {\hfill  June 2018}

\item {Probability in Number Theory}{ \hfill  May 2018}

\item {Arbeitsgemeinschaft in Oberwolfach} {\hfill  Oct 2017}
%\\\item {Additive Combinatorics, Entropy, and Fractal Geometry}

\item {MSRI Summer Graduate School on Automorphic Forms \\and the Langlands Program} {\hfill  Aug 2017}
	%\\\item {Algebraic aspects of Automorphic Forms and representations.}
 
 \item {PCMI Graduate Summer School on random matrices} {\hfill  June 2017}
%\\	{{ Universality of spectral statistics, Free probability, Orthogonal \\polynomials,Concentration methods, Riemann-Hilbert Problems }}
 
\item { University of Houston Summer School on Dynamical Systems} {\hfill  May 2017}
%\\	{{Hyperbolic dynamics, Dynamical methods in Dio\itemantine \\approximation, Dynamics of group actions on homogeneous spaces}}

\item {MSRI: Analytic Number Theory}{\hfill  Jan, May 2017}

\item { West Coast Algebraic Topology Summer School} {\hfill  Aug 2016}
%\\	{{Homotopy theory and number theory}}

\item { Building Bridges: 3rd EU/US Summer School \\and workshop on Automorphic Forms} {\hfill  July 2016}
	%\\{ Explicit L-functions, Automorphic Forms and Galois representations, \\Galois Representations and Dio\itemantine Problems,\\Kronecker limit formalism and moonshine type groups}
 
\item  UNCG Summer School in Computational Number Theory{\hfill  June 2016}
%\\	{ Function Fields } 
 
\item  Houston Summer School on Dynamical Systems{\hfill  May 2016}
%\\	{ Decay of correlations, Hyperbolic dynamics, Multiplicative ergodic \\theory, Poincar\'e sections for diagonal actions}

\item  UNCG Summer School in Computational Number Theory{\hfill  May 2015}
	%\\{ Zeta Functions -- New Theory and Computations} 

\item  Exchange in {University of Wisconsin-Madison}    {\hfill   Fall 2012}
%	
% Summer School in {Nanjing University}     {\hfill  Summer 2012}\\
%	{ Representations of Finite Group and Compact Lie Group,\\ Arithmetic Theory of Elliptic Curves}
%
% Summer School in {Sichuan University} {\hfill  Summer 2012}
%	\\{ Homology Theory and Topics in Geometry}
%	
% Summer School in {Sichuan University}   {\hfill   Summer 2011}
%	\\{ Algebraic Topology and Introduction to Algebraic Geometry}
\end{list1}

\section{\sc Outreach \\Activities }
\begin{list1}
\item Four Color Fest {\hfill  Nov 1-4 2017}
\item A Math Carnival at Illinois-Gathering for Gardener  {\hfill  Jan 28 2017}
\item Science at the Market {\hfill  Aug 2013}
\end{list1}
\section{\sc Skills}

\begin{tabular}{@{}p{0.8in}p{6in}}
	Programming: &  C++, Mathematica, Matlab, Python\\
	Languages:& English, Chinese
\end{tabular}

%\section{\sc Graduate Coursework}
%
%\begin{tabular}{@{}p{2.3in}p{3in}}
%	\begin{list1}
%		\item Abstract Algebra
%		\item Complex Variables
%		\item Differentiable Manifolds
%		\item Analytic Theory of Numbers 
%		\item Algebraic Number Theory
%		\item Algebraic Geometry
%		\item Exponential Sums
%		\item Local fields
%		\item L-functions
%		\item Anatomy of integers and random permutations
%	\end{list1}
%	&
%	\begin{list1}
%		\item Elliptic Curves and Modular Form
%		\item  Theory of Partitions
%		\item Class Field Theory
%		\item Coding Theory
%		\item  Expander Gra\items in Number Theory
%	    \item Correlations and Local Spacing 
%	    \item Mock Theta Functions
%	    \item Anatomy of Intergers
%	    \item Perron-Frobenius Operators
%	\end{list1}
%\end{tabular}



\end{resume}
\end{document}

