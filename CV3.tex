\documentclass[margin,line,pifont,palatino,courier,10pt]{res}
%\documentclass[margin,line,pifont,palatino,courier]{res}

%\usepackage{pifont}
%\usepackage[latin1] {inputenc}
\usepackage[T1]{fontenc}
%\usepackage{palatino}
\usepackage{helvet}
\usepackage{eurosym}
%\topmargin .5in
\oddsidemargin -.3in
\evensidemargin -.3in
\textwidth=5.8in
 \textheight=9.0in
%\itemsep=0in
%\parsep=0in
\usepackage{fancyhdr,hyperref}
%\topmargin=0in
%\textheight=8.5in
\pagestyle{fancy}
\renewcommand{\headrulewidth}{0pt}
\fancyhf{}
%\cfoot{\thepage}
%\lfoot{\textit{\footnotesize Research Statement}}
\rfoot{{\footnotesize \thepage}}


\newenvironment{list1}{
  \begin{list}{$\bullet$}{%
      \setlength{\itemsep}{0in}
      \setlength{\parsep}{0in} \setlength{\parskip}{0in}
      \setlength{\topsep}{0in} \setlength{\partopsep}{0in}
      \setlength{\leftmargin}{0.17in}}}{\end{list}}
\newenvironment{list2}{
  \begin{list}{$-$}{%
      \setlength{\itemsep}{0in}
      \setlength{\parsep}{0in} \setlength{\parskip}{0in}
      \setlength{\topsep}{0in} \setlength{\partopsep}{0in}
      \setlength{\leftmargin}{0.2in}}}{\end{list}}

\begin{document}

\name{Junxian Li \vspace*{.1in}}

\begin{resume}

\section{\sc Contact Information}
%\vspace{.05in}
%\begin{tabular}{@{}p{2.75in}p{2in}}
%	Department of Mathematics                        & \hfill{jli135@illinois.edu}\\
%	University of Illinois at Urbana-Champaign  & \hfill{\href{https://math.illinois.edu/~jli135/}{www.math.illinoi.edu/~jli135}}\\
%	1409 W. Green St.            & \\
%	Urbana, Illinois 61801               & \\
%\end{tabular}
%
%\vspace{.05in}
%\begin{tabular}{@{}p{2.2in}p{3in}}
%Mathematisches Institut                    & \hfill{junxian.li@mathematik.uni-goettingen.de}\\
%Georg-August Universit\"at G\"ottingen
% & \hfill{\href{https://jligit.github.io/}{https://jligit.github.io/}}\\
%Bunsenstra\ss e 3-5
%         & \\
%D-37073 G\"ottingen
%             & \\
%Germany   & 

%\end{tabular}
%\begin{tabular}{@{}p{2.5in}p{2.5in}}
%Universit\"at Bonn               
%& \hfill{jli135@math.uni-bonn.de}\\
%Mathematisches Institut  &  \hfill{\href{https://jligit.github.io/}{https://jligit.github.io/}}\\
%Endenicher Allee 60 & \hfill{she/her/hers} \\
%53115 Bonn, Germany   & 
%\end{tabular}
\vspace{.0in}
Universit\"at Bonn    \hfill{jli135@math.uni-bonn.de}\\
Mathematisches Institut   \hfill{\href{https://jligit.github.io/}{https://jligit.github.io/}}\\
Endenicher Allee 60 \\ %\hfill{she/her/hers} \\
53115 Bonn, Germany   
 


\section{\sc Research Interests}
Analytic Number theory: $L$-functions, Primes,  Exponential sums, Additive Combinatorics\\
Automorphic Forms,
Harmonic analysis, Dynamical systems % Algebraic Curves, 
\section{\sc Employment} 
\textbf{Postdoctoral Researcher } (Mentor: Valentin Blomer) \\
\vspace*{-.15in}
\begin{list1}
\item []{Universit\"at Bonn }  \hfill{2021--Now} 
\item[] {Max Planck Institute for Mathematics}\hfill{ 2019--2021}
\item [] {Georg-August Universit\"at G\"ottingen} \hfill{2018--2019}
\end{list1}

%{Max Planck Institute for Mathematics}\hfill{Sept 2019--Aug 2021}

%\begin{list1}

		%\item[] Mentors: Valentin Blomer and Pieter Moree

%\end{list1}
%{Georg-August Universit\"at G\"ottingen} \hfill{Sept 2018--Aug 2019}

%\begin{list1}

		%\item[] Mentors: Valentin Blomer and Harald Helfgott


%\end{list1}
\section{\sc Education}

\textbf{Ph.D.~in Mathematics}\hfill{2013--2018}\\
\vspace*{-.15in}

\begin{list1}
\item[] University of Illinois at Urbana-Champaign
\item[] Advisor: Alexandru Zaharescu

\end{list1}

\textbf{B. A.~in Mathematics }\hfill{2009--2013}\\
\vspace*{-.15in}
\begin{list1}
\item[] Nanjing University

\end{list1}






\section{\sc Publications}
%21. {A two dimensional version of the delta symbol method} (with S. Rydin Myerson and P. Vishe), preprint, 30 pp.

20. {Simultaneous large values and dependence of Dirichlet L-functions in the critical strip} (with S. Inoue), arXiv:2211.15165, 23 pp. 

19. {Correlations of values of random diagonal forms} (with V. Blomer), arXiv:2211.02469, 28 pp.


18. {Additive problems with almost prime squares} (with V. Blomer, L.  Grimmelt and S. Rydin Myerson), arXiv:2111.01601, 50 pp. %{\it under consideration in Compos. Math. } 40 pp.

17. {Joint value distribution of $L$-functions on the critical line} (with S. Inoue), arXiv:2102.12724, 50 pp. % {\it under consideration in Math. Z. } 50 pp.

16. {Lower bounds for discrete negative moments of the Riemann zeta function (with W. Heap and J. Zhao)}, \textbf{  Algebra Number Theory} \emph{16} (2022), no. 7, 1589-1625.% 40 pp.%arXiv:2003.09368

15. {Large values of Dirichlet $L$-functions at zeros of a class of $L$-functions}, \textbf{Canad. J. Math. } \emph{ 73} (2021), no. 6, 1459--1505.

14. {Uniform Titchmarsh divisor problems} (with E. Assing and V. Blomer), \textbf{Adv. Math.} \emph{393} (2021), Paper No. 108076, 51 pp. % arXiv:2005.13915

13. {Leading Digits of Mersenne Numbers} (with Z. Cai, M Faust, A. J. Hildebrand, and Y. Zhang), \textbf{ Exp. Math.}  \emph{30} (2021), no. 3, 405--421. %arXiv:1712.04425.

12. {The surprising accuracy of Benford's law in mathematics} (with Z. Cai, M. Faust, A. J. Hildebrand and Y. Zhang), \textbf{ Amer. Math. Monthly}  \emph{127} (2020), no. 3, 217--237.
 (\emph{winner of the} \textbf{Paul R. Halmos-Lester R. Ford Award}) 

11. {Ducci iterates and similar ordering on sets of visible points} (with A. Tamazyan and A. Zaharescu), \textbf{ Int. J. Number Theory} \emph{16} (2020), no. 1, 1--28.

10. {A binary quadratic Titchmarsh divisor problem}
\textbf{ Acta Arithmetica} \emph{192} (2020), no. 4, 341--361. %arXiv:1808.00837

9. {The final problem: an identity from Ramanujan's lost notebook} (with B. Berndt and A. Zaharescu), \textbf{ J. Lond. Math. Soc.} (2) \emph{100} (2019), no. 2, 568--591.

8. {Almost Beatty Partitions} (with A. J. Hildebrand, X. Li, and Y. Xie), {\textbf{ J. Integer Seq.} \emph{22} (2019), no. 4, Art. 19.4.6, 34 pp.}
%arXiv:1809.08690 

7. {Value distribution of $L'(\rho)$} (with A. Zaharescu), {\textbf{ J. Math. Anal. Appl.} \emph{480} (2019), no. 1, 123400, 24 pp.}

6. {A local Benford Law for a class of arithmetic sequences} (with Z. Cai and A. J. Hildebrand), \textbf{ Int. J. Number Theory} \emph{15} (2019), no. 3, 613--638.
%arXiv:1808.01496

5. {On distinct consecutive $r$-differences} (with G. Shakan), \textbf{ J. Number Theory} \emph{199} (2019), 363--376.
% arXiv:1708.03742.	

4. {Exact evaluation of second moments associated with some families of
	curves over a finite field} (with R. Donepudi and A. Zaharescu), \textbf{ Finite Fields Appl.} \emph{48} (2017), 331--355.

3. {A lower bound for the least prime in an arithmetic progression} (with K. Pratt and G. Shakan), \textbf{ Q. J. Math.} \emph{68} (2017), no. 3, 729--758.

2. {Smooth {$L^2$} distances and zeros of approximations of {D}edekind zeta functions} (with M. Nastasescu, A. Roy, and A. Zaharescu), \textbf{ Manuscripta Math.}  \emph{154} (2017), no. 1-2, 195--223.

1. {Zeros of a family of approximations of {H}ecke {$L$}-functions
	associated with cusp forms} (with A. Roy and A. Zaharescu), \textbf{ Ramanujan J.} \emph{41} (2016), no. 1-3, 391--419.


\section{\sc Conference Proceedings}
2. {The Final Problem: A Series Identity from the Lost Notebook} (with B. C. Bruce and A. Zaharescu),  \emph{ George E.~Andrews 80 Years of Combinatory Analysis}, K.~Alladi, B.~C.~Berndt, P.~Paule, J.~Sellers, and A.~J.~Yee, eds.,  Birkh\"{a}user, 783--790, 2021.

1. {On primes in arithmetic progressions}
%\newblock {
%	Building Bridges 3 conference proceedings to appear.}
\newblock{{Automorphic forms and related topics}, 165--167, {\it Contemp. Math.} 732, Amer. Math. Soc., Providence, RI, 2019.} 




\section{\sc Honors and Awards}
\begin{list1}
	\item \textbf{The Paul R. Halmos-Lester R. Ford Award} \hfill{2021}
	
	\begin{list1}
		\item[]	for outstanding expository papers in The American Mathematical Monthly
	\end{list1}
\end{list1}


\begin{list1}
\item \textbf{Bateman Fellowship }  \hfill{2018}

\begin{list1}
	\item[]	for excellence in Number Theory
\end{list1}
\end{list1}

\begin{list1}
	\item \textbf{On the List of Teachers Ranked as Excellent by their Students} \hfill{Fall 2017}
\end{list1}






%\section{\sc Grants}
%\begin{list1}
%\item	\textbf{Conference Grants} 
%	\begin{list2}
%		\item Bonn International Graduate school of Mathematics	{(\euro 11,650)} \hfill{ Mar 2022}
%	\end{list2}
%\end{list1}
%\begin{list1}
%\item \textbf{Travel Grants}
%\begin{list2}
%	\item{AMS-Mathematics Research Communities Grant for the JMM } \hfill{   Jan 2019}
%\item	{US Junior Oberwolfach Fellows (NSF Grant)} \hfill{   Oct 2017}
%	\item{PCMI-Graduate Summer School Travel Award } \hfill{   June 2017}
%\item	{UIUC-Association for Women in Mathematics Graduate Travel Funding } \hfill{   2017--2018}
%\item	{AMS Graduate Student Travel Grant for the JMM} \hfill{  Jan 2017}
%	
%\end{list2}
%\end{list1}


\section{\sc Teaching}
\begin{list1}
	\item \textbf{Graduate Courses and Seminars}
	\begin{list2}
		\item Topic course: Sieve Methods, Instructor {\hfill  Bonn,  2022}
		\item Number Theory Learning Seminar {\hfill  G\"ottingen,  2018-2019}
	\end{list2}
\end{list1}
\begin{list1}
	\item \textbf{Undergraduate Courses}
	\begin{list2}
\item Math 415 Linear Algebra, Instructor {\hfill  UIUC, 2017}
%Math 415 Linear Algebra, Instructor {\hfill  UIUC, Spring 2017}\\
\item Math 231 Calculus II, Instructor {\hfill  UIUC,  2016}
\item Math 241 Calculus III, Instructor {\hfill  UIUC, 2016--2015} 
%Math 241 Calculus III, Instructor {\hfill  UIUC, }
	\end{list2}
\end{list1}



\section{\sc Student \\Mentoring}
\begin{list1}
	\item \textbf{Master thesis supervision}
	\begin{list2}
		\item Ivan Chan Kai Chin (Universit\"at Bonn) {\hfill 2022-}
	\end{list2}
\end{list1}
\begin{list1}
	\item \textbf{Undergraduate Student Mentoring in Illinois Geometry Lab (10 projects)}
	\begin{list2}
		\item Almost Beatty Partitions;  Beatty sequences, and Partitions of the Integers {\hfill   2018}
		\item Chaotic maps and exotic number systems{\hfill  Fall 2017}
		\item Finding integers in group orbits {\hfill  Spring 2017}
		\item Local Benford's Law; Leading digit distribution {\hfill  2016} 
	%	\item Random Walk in number theory {\hfill  Fall 2015} 
		\item Fractals, Patterns and Randomness in Number Theory {\hfill 2015} 
		\item Fourier Series with Number theoretic coefficients {\hfill  Fall 2014} 
		\item Symmetry in Nature {\hfill Spring 2014}
	\end{list2}
\end{list1}

\section{\sc Professional \\Services}

\begin{list1}
	\item \textbf{Organizer of a workshop at Universit\"at Bonn} {\hfill  2022}
	
	\begin{list2}
		\item Young Scholars in the Analytic Theory of
		Numbers and Automorphic Forms
	\end{list2}
	
	\item \textbf{Organizer of AMS Special Session at the Joint Mathematics Meeting} {\hfill   2019}
	
	\begin{list2}
		\item Number Theoretic Methods in Hyperbolic Geometry 
	\end{list2}
	\item \textbf{Organizer of Graduate Student Number Theory Seminar in UIUC} {\hfill 2016--2018}
	\item \textbf{Referee}: 
	\begin{list2}
		\item Canad. J. Math.;  Res. Number Theory; Monatsh. Math. \item Ramanujan J.; J. Number Theory
		\item Math. Reports; Rev. Roumaine Math. Pures Appl.;J. Math. Sci. Adv. Appl.
	\end{list2}
	%\item Membership: American Mathematical Society
	
\end{list1}

\section{\sc Invited Seminar Talks and \\Conferences }

\begin{list1}
\item \textbf{Two dimensional Kloosterman's refinement of the circle method}
		\begin{list2}
			\item {Number Theory Seminar, Lille }{\hfill Sept 2022}
			\item {Oberseminar Analytic Number Theory and Automorphic Forms, Bonn }{\hfill Apr 2022}
		\end{list2}
\item \textbf{Hardy-Littlewood problems with almost primes}
\begin{list2}
	\item {Analytic Number Theory Meetings, IHP}. {\hfill Sept 2022}
	\item {Number Theory Days, HKU (online)}. {\hfill July 2022}
	\item {Workshop in Number theory and Harmonic Analysis, SDU (online)}. {\hfill July 2022}
	\item {Number Theory Seminar, UIUC (online)}. {\hfill Mar 2022}
	\item {Heilbronn Number Theory Seminar, Bristol (online)}. {\hfill Jan 2022}
	\item {Number Theory Seminar, XJTU (online)}. {\hfill Dec 2021}
\end{list2}

\item \textbf{Simultaneous large values of Dirichlet L-functions in the critical strip}
\begin{list2}
	\item {Oberseminar Analytic Number Theory and Automorphic Forms, Bonn}. {\hfill Oct 2021}
\end{list2}
\item \textbf{Joint Value distribution of $L$-functions}
\begin{list2}
	\item {Number Theory Seminar, PIMS Collaborative Research Group (online)}. {\hfill Sept 2022}
	\item {Qilu Youth Forum, SDU (online)}. {\hfill Sept 2021}
    \item {Number theory lunch seminar, MPIM (online)}. {\hfill Sept 2021}
\end{list2} 
\item \textbf{Uniform Titchmarsh Divisor Problems}
\begin{list2}
	\item {S\'eminaire ADA, Calais}. {\hfill Sept 2022}
	\item {Number theory Seminar, SDU (online)}. {\hfill May 2021}
	\item {PIMS-Lethbridge Number Theory Seminar, Lethbridge (online)}. {\hfill Mar 2021}
	\item {Japan Europe Number Theory Exchange Seminar}. {\hfill Jan 2021}
\end{list2} 
\item \textbf{ Joint Value Distribution of $L$-functions.} 
\begin{list2}
	\item {Oberseminar Analytic Number Theory, Bonn (online)}.{\hfill Nov 2020}
\end{list2}
\item \textbf{ Derivative of the Riemann zeta function at its zeros.}
\begin{list2}
	 \item {Analytic Number Theory Meeting, IHP (online)}.{\hfill June 2020}
	 
\end{list2}

\item \textbf{Extreme values of $L$-functions}
\begin{list2}
	\item { Number theory lunch seminar, MPIM}.{\hfill Oct 2019}
	\item {Oberseminar number theory, Georg-August Universit\"at G\"ottingen}.{\hfill Nov 2018}
\end{list2}
 \item \textbf{The Unreasonable Effectiveness of Benford's Law in Mathematics}
 \begin{list2}
 	\item Joint with A. J. Hildebrand, Number Theory Seminar, UIUC.{\hfill Apr 2018}
 \end{list2}
 
\item \textbf{Primes in arithmetic progressions} 
\begin{list2}
	\item{Junior Mathematics Colloquium, Georg-August Universit\"at G\"ottingen}.{\hfill Dec 2017}
\end{list2}

\item \textbf{Randomness in Number Theory} 
\begin{list2}
	\item {Graduate Student Colloquium, UIUC}.{\hfill Nov 2017}
\end{list2}


\item \textbf{Primes in arithmetic progressions} 
\begin{list2}
	\item {Where Geometry meets Number Theory, a conference in honor of \\the 60th birthday of Per Salberger, Gothenburg}.{\hfill July 2017}
\end{list2}
	
 \item \textbf{The least prime in an arithmetic progression}
 	\begin{list2}
 		\item Joint Mathematics Meeting, Atlanta. {\hfill Jan 2017}
 		\item Number Theory Seminar, UIUC.{\hfill Sept 2016}
 		\item Workshop on Automorphic Forms and Related Topics, Sarajevo . {\hfill Jul 2016}
 	\end{list2}
\item \textbf{Approximations of $L$-functions} 
 \begin{list2}
 	\item  Midwest Number Theory Conference for Graduate Students \\and Recent Ph.D's. {\hfill Oct 2015}
 	\item Graduate Student Number Theory Seminar, UIUC.{\hfill Nov 2015}
 \end{list2} 
 \item \textbf{ Bailey Pairs and Bailey chains} 
 \begin{list2}
 	\item $q$-series  Seminar, UIUC.{\hfill Apr 2015}
 \end{list2}

 \item \textbf{ Basic Hypergoemetric functions} 
 \begin{list2}
 	\item $q$-series  Seminar, UIUC.{\hfill Mar 2015}
 \end{list2}
\end{list1}

\section{\sc Conferences\\ and Summer schools}
\begin{list1}
\item {Analytic Number Theory Workshop, Oberwolfach} \hfill{Nov 2022}

\item {50 years of Number Theory and Random Matrix Theory, IAS} \hfill{June 2022}

\item {Harmonic Analysis and Number Theory, ETH} \hfill{Mar 2022}

\item {Zeta functions, CIRM}{\hfill Dec 2019}

\item {Second Symposium on Analytic Number Theory, Cetraro}{\hfill July 2019}

\item {Rational points on irrational varieties, IHP}{\hfill June 2019}

\item {L-functions and Multiplicative Number Theory, U of Mississippi}{\hfill May 2019}

\item {Distribution of values of zeta functions and L-functions, RIKEN}{\hfill Mar 2019}

\item {Workshop and Winter School on Local Statistics of Point Sequences, Linz}{\hfill Feb 2019}

\item {Building Bridges: 4th EU/US Summer School \\and Workshop on Automorphic Forms and Related Topics
} {\hfill  July 2018}
%\item{Computational aspects of Shimura Curves,\\The resolvent kernel and the Maass-Shimura-Shintani lifting,\\ Theta liftings and modularity of abelian varieties}

\item {Hausdorff School: L-functions: Open Problems and Current Methods}{\hfill  June 2018}

\item {MRC: Number Theoretic Methods
	in Hyperbolic Geometry
	} {\hfill  June 2018}

\item {Probability in Number Theory}{ \hfill  May 2018}

\item {Arbeitsgemeinschaft in Oberwolfach} {\hfill  Oct 2017}
%\\\item {Additive Combinatorics, Entropy, and Fractal Geometry}

\item {MSRI Summer Graduate School on Automorphic Forms \\and the Langlands Program} {\hfill  Aug 2017}
	%\\\item {Algebraic aspects of Automorphic Forms and representations.}
 
 \item {PCMI Graduate Summer School on random matrices} {\hfill  June 2017}
%\\	{{ Universality of spectral statistics, Free probability, Orthogonal \\polynomials,Concentration methods, Riemann-Hilbert Problems }}
 
\item { University of Houston Summer School on Dynamical Systems} {\hfill  May 2017}
%\\	{{Hyperbolic dynamics, Dynamical methods in Dio\itemantine \\approximation, Dynamics of group actions on homogeneous spaces}}

\item {MSRI: Analytic Number Theory}{\hfill  Jan, May 2017}

\item { West Coast Algebraic Topology Summer School} {\hfill  Aug 2016}
%\\	{{Homotopy theory and number theory}}

\item { Building Bridges: 3rd EU/US Summer School \\and workshop on Automorphic Forms} {\hfill  July 2016}
	%\\{ Explicit L-functions, Automorphic Forms and Galois representations, \\Galois Representations and Dio\itemantine Problems,\\Kronecker limit formalism and moonshine type groups}
 
\item  UNCG Summer School in Computational Number Theory{\hfill  June 2016}
%\\	{ Function Fields } 
 
\item  Houston Summer School on Dynamical Systems{\hfill  May 2016}
%\\	{ Decay of correlations, Hyperbolic dynamics, Multiplicative ergodic \\theory, Poincar\'e sections for diagonal actions}

\item  UNCG Summer School in Computational Number Theory{\hfill  May 2015}
	%\\{ Zeta Functions -- New Theory and Computations} 

\item  Exchange in {University of Wisconsin-Madison}    {\hfill   Fall 2012}
%	
% Summer School in {Nanjing University}     {\hfill  Summer 2012}\\
%	{ Representations of Finite Group and Compact Lie Group,\\ Arithmetic Theory of Elliptic Curves}
%
% Summer School in {Sichuan University} {\hfill  Summer 2012}
%	\\{ Homology Theory and Topics in Geometry}
%	
% Summer School in {Sichuan University}   {\hfill   Summer 2011}
%	\\{ Algebraic Topology and Introduction to Algebraic Geometry}
\end{list1}

\section{\sc Outreach \\Activities }
\begin{list1}
\item Four Color Fest {\hfill  Nov 1-4 2017}
\item A Math Carnival at Illinois-Gathering for Gardener  {\hfill  Jan 28 2017}
\item Science at the Market {\hfill  Aug 2013}
\end{list1}
\section{\sc Skills}

\begin{tabular}{@{}p{0.8in}p{6in}}
	Programming: &  C++, Mathematica, Matlab, Python\\
	Languages:& Chinese, English
\end{tabular}

%\section{\sc Graduate Coursework}
%
%\begin{tabular}{@{}p{2.3in}p{3in}}
%	\begin{list1}
%		\item Abstract Algebra
%		\item Complex Variables
%		\item Differentiable Manifolds
%		\item Analytic Theory of Numbers 
%		\item Algebraic Number Theory
%		\item Algebraic Geometry
%		\item Exponential Sums
%		\item Local fields
%		\item L-functions
%		\item Anatomy of integers and random permutations
%	\end{list1}
%	&
%	\begin{list1}
%		\item Elliptic Curves and Modular Form
%		\item  Theory of Partitions
%		\item Class Field Theory
%		\item Coding Theory
%		\item  Expander Gra\items in Number Theory
%	    \item Correlations and Local Spacing 
%	    \item Mock Theta Functions
%	    \item Anatomy of Intergers
%	    \item Perron-Frobenius Operators
%	\end{list1}
%\end{tabular}



\end{resume}
\end{document}

